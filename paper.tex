% This is samplepaper.tex, a sample chapter demonstrating the
% LLNCS macro package for Springer Computer Science proceedings;
% Version 2.21 of 2022/01/12
%
\documentclass[runningheads]{llncs}
%
\usepackage[T1]{fontenc}
% T1 fonts will be used to generate the final print and online PDFs,
% so please use T1 fonts in your manuscript whenever possible.
% Other font encondings may result in incorrect characters.
%
\usepackage{graphicx}
% Used for displaying a sample figure. If possible, figure files should
% be included in EPS format.
%
% If you use the hyperref package, please uncomment the following two lines
% to display URLs in blue roman font according to Springer's eBook style:
%\usepackage{color}
%\renewcommand\UrlFont{\color{blue}\rmfamily}
%\urlstyle{rm}
%
\begin{document}
%
\title{%
AI in Halthcare and LGPD -- Federated Learning with Differential Privacy \\
\large An introductory survey on solutions for training models under privacy and regulatory constraints
}
%
%\titlerunning{Abbreviated paper title}
% If the paper title is too long for the running head, you can set
% an abbreviated paper title here
%
\author{André Willyan de S. Vital\inst{1}\orcidID{537550}\and
Edson Coelho\inst{1}\orcidID{matrícula} \and
Israel Nícolas\inst{1}\orcidID{matrícula}}
%
\authorrunning{F. Author et al.}
% First names are abbreviated in the running head.
% If there are more than two authors, 'et al.' is used.
%
\institute{Universidade Federal do Ceará}
%
\maketitle              % typeset the header of the contribution
%
\begin{abstract}
ABSTRACT HERE
\keywords{("Federated Learning" OR "FL") AND ("Differential Privacy" OR "DP") AND ("Healthcare" OR "Medical Data" OR "EHR" OR "Electronic Health Records")}
\end{abstract}

\section{Introduction}
\section{Related Works}
\section{Preliminaries}
\section{Taxonomy}
\section{State of the Art}
\section{Comparative Analysis}
\section{Open Challenges and Research Directions}
\section{Best Practices and Recommendations}
\section{Conclusion}


%
% ---- Bibliography ----
%
% BibTeX users should specify bibliography style 'splncs04'.
% References will then be sorted and formatted in the correct style.
%
% \bibliographystyle{splncs04}
% \bibliography{mybibliography}
%
\end{document}
